\documentclass[a4paper,12pt]{article}

\usepackage{longtable}
\usepackage{enumitem}[shortlabels]					% Para personalizar listas
\usepackage{dirtytalk}

\usepackage[spanish, mexico]{babel}
    \decimalpoint
\usepackage[colorlinks,linkcolor=black,urlcolor=black,citecolor=black, breaklinks=true]{hyperref}
\usepackage{fancyhdr}
\usepackage{tikz}
\usetikzlibrary{arrows}
\usetikzlibrary{patterns}
\usetikzlibrary{shapes}
\usetikzlibrary{positioning}
\usetikzlibrary{automata}
\usetikzlibrary{cd}
\usepackage{tikz-3dplot}
\usepackage{xcolor}
\usepackage[utf8]{inputenc}
\usepackage[T1]{fontenc}
\usepackage[intlimits]{amsmath}
\usepackage{fullpage}
% \usepackage[osf,sc]{mathpazo} % Uncomment for Palatino and comment out the next line
\usepackage[frenchstyle,widermath,narrowiints,fullsumlimits,fullintlimits]{kpfonts} % Comment out and uncomment the previous line for Palatino
\linespread{1.05}
\usepackage{amsfonts}
\usepackage{amsthm}
\usepackage{amsxtra}
\usepackage{amssymb}
\usepackage{mathdots}
\usepackage{mathrsfs}
\usepackage{microtype}
\usepackage{stmaryrd}
\usepackage{titlesec}
\usepackage{systeme}
\usepackage[titles]{tocloft}
\usepackage{textcase}
\usepackage{setspace}
\usepackage{xfrac}
\usepackage{mathtools}
\usepackage{faktor}
\usepackage{cancel}
\usepackage{mparhack}
\usepackage{booktabs}
\usepackage{multirow}
%\usepackage[fixlanguage]{babelbib}
\usepackage{tikz}
\usepackage{csquotes}
\usepackage{forest}
\usepackage{xparse}
\usepackage{ytableau} %Diagramas de Young
\usepackage[
            hyperref=true, % También puede ser auto
            style=nature
            ]{biblatex}
\addbibresource{biblio.bib}
\usepackage{nicematrix}
\usepackage{macros_math}
\NiceMatrixOptions{cell-space-limits = 2pt}
% paquetes para obtener el 1 bomnito
\usepackage[bb=dsserif]{mathalpha}
\usepackage{bm}
\newcommand\Bbbbone{%
  \ifdefined\mathbbb%
    \mathbbb{1}%
  \else%
    \boldsymbol{\mathbb{1}}%
  \fi}

\pgfdeclarelayer{bg}
\pgfsetlayers{bg,main}

\tikzset{
  wrap/.style={
    line cap=round,
    #1,
    line width=21pt,
    opacity=0.3,
  },
  mynode/.style={
    draw,
    circle,
    yshift=-0.5cm,
    outer sep=0.3cm
  },
  group/.style={
    draw,
    ellipse,
    minimum width=3cm,
    minimum height=6cm
  },
}


\usepackage[spanish]{cleveref} 

\renewcommand{\labelenumi}{(\alph{enumi})}
\title{Tarea 3 --- Procesos Estocásticos}
\author{
Pavón Alvarez, Lenin
%\and 
}
\date{Otoño 2023}
%
\allowdisplaybreaks
\begin{document}
\maketitle
\section{Problema 1}
\begin{displayquote}
    Vuelve a resolver la primera parte del examen parcial 1
\end{displayquote}
\subsection{Reactivo 1}
\begin{displayquote}
    Sea $(\Omega, \mathcal{F}, \p)$ un espacio de probabilidad. Definido en dicho espacio, sea $X=\cdm[n][0]{X_n}$ una cadena de Marjov con espacio de estados $E$.
    \begin{enumerate}
        \item Escribe (formalmente) la propiedad de Markov para $\cdm[n][0]{X_n}$
        \item Explica intuitivamente el significado de la propiedad de Markov
    \end{enumerate}
\end{displayquote}
\subsection{Reactivo 2}
\begin{displayquote}
    Demuestra que todo estado absorbente es recurrente.
\end{displayquote}
\subsection{Reactivo 3}
\begin{displayquote}
    Considera la cadena de Markov con espacio de estados $\{1,2,3,4,5\}$ y matriz de transición
    \[\begin{pNiceMatrix}
        0&0&0&\frac{1}{2}&\frac{1}{2}\\
        \frac{1}{2}&0&\frac{1}{8}&\frac{1}{4}&\frac{1}{8}\\
        0&0&1&0&0\\
        0&0&0&\frac{1}{2}&\frac{1}{2}\\
        0&0&0&\frac{1}{3}&\frac{2}{3}
    \end{pNiceMatrix}\]
\end{displayquote}
\begin{enumerate}
    \item Dibuja su diagrama de comunicación
    \item Encuentra las clases de comunicación
    \item Lista las clases de comunicación cerradas
    \item Determina estados absorbentes (si existen).
    \item Para cada estado, determina si es recurrente o transitorio. \textit{Justifica tu respuesta}
    \item Calcula $\p(X_2=4|X_0=1)$
\end{enumerate}
\subsection{Reactivo 4}
\begin{displayquote}
    Sea $X=X=\cdm[n][0]{X_n}$ una cadena de Marjov con espacio de estados $E$ y probabilidades de transición $(P_{x,y})_{x,y\in E}$. Sea $y\in E$ y consideremos como distribución inicial $\delta_y$ (es decir, $X_0=y$) con probabilidad $1$
    \par Supongamos que $y$ es un estado transitorio y consideremos la variable aleatoria
    \[V_y=\sum_{n=0}^\infty\uno(X_n=y)\]
    que corresponde al \textit{número de visitas} al estado $y$. Determian la distribución de $V_y$, indicando el nombre de la distribución y su parámetro.
\end{displayquote}
\section{Problema 2}
\begin{displayquote}
    Una pulga amaestrada brinca aleatoriamente en los vértices de un triángulo, todos sus saltos tienen la misma probabilidad (se supone que en cada etapa cambia de lugar).
    \begin{enumerate}
        \item Calcula la probabilidad de que después de $n$ saltos la pulga regrese a su punto de partida
        \item Suponga que otra pulga brinca en los vértices de un triángulo, pero que esta brinca el doble de veces a su derecha que a su izquierda. ¿Cuál es la probabilidad de que después de $n$ saltos la pulga regrese a su punto de partida?
    \end{enumerate}
\end{displayquote}
\section{Problema 3}
\begin{displayquote}
    Supongamos que la prbabilidad de que llueva hoy es de $0.3$ si ninguno de los últimos dos días fue lluvioso, pero la probabilidad de que llueva hoy es $0.6$ si al menos uno de los últimos dos días fue lluvioso.
    \par Indicamos con $W_n$ el clima del $n-$ésimo día y denotamos
    \[W_n=\begin{cases}
        R& \text{si en el $n$-ésimo día llovió,}\\
        S& \text{si el clima del $n$-ésimo día fue soleado}
    \end{cases}\]
    \begin{enumerate}
        \item Muestra que $W=\cdm{W_n}$ no es una cadena de Markow
        \item Considera el proceso estocástico $X=\cdm{X_n}$ definido por
        \[X_n= (W_{n-1},W_n),\qquad n\geqslant 1\]
        Demuestra que $X$ es una cadena de Markov con espacio de estados 
        \[\{(R,R),(R,S),(S,R),(S,S)\}\]
        \item Calcula la matriz de transición de $X$
        \item Calcula las pribabilidades de transición a dos pasos de $X$
        \item ¿Cuál es la probabilidad de que sí llueva un miércoles dado que no llovió el domineog ni el lunes?
    \end{enumerate}
\end{displayquote}
\section{Problema 4}
Seis niños (Dick, Helen, Joni, Mark,
Sam y Tony) juegan a atrapar la pelota y sucede lo siguiente:
\begin{itemize}
    \item Si Dick tiene la pelota, es igualmente probable que se la lance a Helen, Mark, Sam y Tony.
    \item Si Helen tiene la pelota, es igualmente probable que se la lance a Dick, Joni, Sam y Tony.
    \item Si Sam tiene la pelota, es igualmente probable que se la lance a Dick, Helen, Mark y Tony.
    \item Si Joni o Tony reciben la pelota, siguen lanzándosela el uno al otro.
    \item Si Mark recibe el balón, se escapa con él.
\end{itemize}
\begin{enumerate}
    \item Encuentra la matriz de transición de la cadena de Markov que modela este juego
    \item Determina las clases de comunicación de la cadena de Markov y determina cuáles clases son cerradas
    \item Para cada estado, determina si es transtiorio o recurrente
    \item Para cada estado, determina si es absorbente
    \item Supongamos que Dick tiene la pelota al comienzo del juego. ¿Cuáles la probabilidad de que Mark termine con la pelota?
\end{enumerate}


\end{document}
