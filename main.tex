\documentclass[a4paper,12pt]{article}

\usepackage{longtable}
\usepackage{enumitem}[shortlabels]					% Para personalizar listas
\usepackage{dirtytalk}

\usepackage[spanish, mexico]{babel}
    \decimalpoint
\usepackage[colorlinks,linkcolor=black,urlcolor=black,citecolor=black, breaklinks=true]{hyperref}
\usepackage{fancyhdr}
\usepackage{tikz}
\usetikzlibrary{arrows}
\usetikzlibrary{patterns}
\usetikzlibrary{shapes}
\usetikzlibrary{positioning}
\usetikzlibrary{automata}
\usetikzlibrary{cd}
\usepackage{tikz-3dplot}
\usepackage{xcolor}
\usepackage[utf8]{inputenc}
\usepackage[T1]{fontenc}
\usepackage[intlimits]{amsmath}
\usepackage{fullpage}
% \usepackage[osf,sc]{mathpazo} % Uncomment for Palatino and comment out the next line
\usepackage[frenchstyle,widermath,narrowiints,fullsumlimits,fullintlimits]{kpfonts} % Comment out and uncomment the previous line for Palatino
\linespread{1.05}
\usepackage{amsfonts}
\usepackage{amsthm}
\usepackage{amsxtra}
\usepackage{amssymb}
\usepackage{mathdots}
\usepackage{mathrsfs}
\usepackage{microtype}
\usepackage{stmaryrd}
\usepackage{titlesec}
\usepackage{systeme}
\usepackage[titles]{tocloft}
\usepackage{textcase}
\usepackage{setspace}
\usepackage{xfrac}
\usepackage{mathtools}
\usepackage{faktor}
\usepackage{cancel}
\usepackage{mparhack}
\usepackage{booktabs}
\usepackage{multirow}
%\usepackage[fixlanguage]{babelbib}
\usepackage{tikz}
\usepackage{csquotes}
\usepackage{forest}
\usepackage{xparse}
\usepackage{ytableau} %Diagramas de Young
\usepackage[
            hyperref=true, % También puede ser auto
            style=nature
            ]{biblatex}
\addbibresource{biblio.bib}
\usepackage{nicematrix}
\usepackage{macros_math}
\NiceMatrixOptions{cell-space-limits = 2pt}
% paquetes para obtener el 1 bomnito
\usepackage[bb=dsserif]{mathalpha}
\usepackage{bm}
\newcommand\Bbbbone{%
  \ifdefined\mathbbb%
    \mathbbb{1}%
  \else%
    \boldsymbol{\mathbb{1}}%
  \fi}

\pgfdeclarelayer{bg}
\pgfsetlayers{bg,main}

\tikzset{
  wrap/.style={
    line cap=round,
    #1,
    line width=21pt,
    opacity=0.3,
  },
  mynode/.style={
    draw,
    circle,
    yshift=-0.5cm,
    outer sep=0.3cm
  },
  group/.style={
    draw,
    ellipse,
    minimum width=3cm,
    minimum height=6cm
  },
}


\usepackage[spanish]{cleveref} 

\renewcommand{\labelenumi}{(\alph{enumi})}
\title{Tarea 3 --- Procesos Estocásticos}
\author{
Pavón Alvarez, Lenin
%\and 
}
\date{Otoño 2023}
%
\allowdisplaybreaks
\begin{document}
\maketitle
\section{Problema 1}
\begin{displayquote}
    Vuelve a resolver la primera parte del examen parcial 1
\end{displayquote}
\subsection{Reactivo 1}
\begin{displayquote}
    Sea $(\Omega, \mathcal{F}, \p)$ un espacio de probabilidad. Definido en dicho espacio, sea $X=\cdm[n][0]{X_n}$ una cadena de Marjov con espacio de estados $E$.
    \begin{enumerate}
        \item Escribe (formalmente) la propiedad de Markov para $\cdm[n][0]{X_n}$
        \item Explica intuitivamente el significado de la propiedad de Markov
    \end{enumerate}
\end{displayquote}
\subsection{Reactivo 2}
\begin{displayquote}
    Demuestra que todo estado absorbente es recurrente.
\end{displayquote}
\subsection{Reactivo 3}
\begin{displayquote}
    Considera la cadena de Markov con espacio de estados $\{1,2,3,4,5\}$ y matriz de transición
    \[\begin{pNiceMatrix}
        0&0&0&\frac{1}{2}&\frac{1}{2}\\
        \frac{1}{2}&0&\frac{1}{8}&\frac{1}{4}&\frac{1}{8}\\
        0&0&1&0&0\\
        0&0&0&\frac{1}{2}&\frac{1}{2}\\
        0&0&0&\frac{1}{3}&\frac{2}{3}
    \end{pNiceMatrix}\]
\end{displayquote}
\begin{enumerate}
    \item Dibuja su diagrama de comunicación
    \item Encuentra las clases de comunicación
    \item Lista las clases de comunicación cerradas
    \item Determina estados absorbentes (si existen).
    \item Para cada estado, determina si es recurrente o transitorio. \textit{Justifica tu respuesta}
    \item Calcula $\p(X_2=4|X_0=1)$
\end{enumerate}
\paragraph{a)} El diagrama de comunicación

\begin{figure}[!hbt]
\centering
\begin{tikzpicture}[->, >=stealth,shorten >=2pt, line width=0.5pt, node distance=2cm]
% Nodos
\node [circle , draw , text opacity=1, text=black] (A)  at  (90:5cm) {$1$};
\node [circle, fill=gray!10,text opacity=1, text=black] (B)  at (162:5cm) {$2$};
\node [circle, fill=gray!30 , draw, opacity=0.5 ,dotted, text opacity=1, text=black] (C)  at (234:5cm) {$3$};
\node [circle , fill=gray!60 , draw , dashed , text opacity=1, text=black] (D)  at (306:5cm) {$4$};
\node [circle , fill=gray!60 , draw , dashed , text opacity=1, text=black] (E)  at  (18:5cm) {$5$};
% Ejes
% ---
% A
% ---
%\path (A) edge [loop above] node [above] {$0.4$} (A);

%\path (A) edge [bend left] node [fill=white, anchor=center, pos=0.3] {$0.3$} (B);

%\path (A) edge [bend left] node [fill=white, anchor=center, pos=0.7] {$0.3$} (C);

\path (A) edge [bend left] node [fill=white, anchor=center, pos = 0.15] {$\frac{1}{2}$} (D);

\path (A) edge [bend left] node [above right] {$\frac{1}{2}$} (E);

% ---
% B
% ---
\path (B) edge [bend left] node [fill=white, anchor=center, pos = 0.25] {$\frac{1}{2}$} (A);

%\path (B) edge [in=132, out=162, loop] node [above] {$0.5$} (B);

\path (B) edge [bend left] node [right] {$\frac{1}{8}$} (C);

\path (B) edge [bend left] node [fill=white, anchor=center, pos=0.5] {$\frac{1}{4}$} (D);

\path (B) edge [bend left] node [fill=white, anchor=center, pos = 0.25] {$\frac{1}{8}$} (E);
% ---
% C
% ---
%\path (C) edge [bend left] node [fill=white, anchor=center, pos=0.15] {$0.5$} (A);

%\path (C) edge [bend left] node [right] {$0.3$} (B);

\path (C) edge [in=204, out=234, loop] node [below] {$1$} (C);

%\path (C) edge [bend left] node [right] {$0.0$} (D);

%\path (C) edge [bend left] node [above right] {$0.0$} (E);
% ---
% D
% ---
%\path (D) edge [bend left] node [below] {$0.3$} (A);

%\path (D) edge [bend left] node [fill=white, anchor=center, pos=0.25] {$0.5$} (B);

%\path (D) edge [bend left] node [right] {$0.0$} (C);

\path (D) edge [in=276, out=306, loop] node [below] {$\frac{1}{2}$} (D);

\path (D) edge [bend left] node [fill=white, anchor=center, pos=0.25] {$\frac{1}{2}$} (E);
% ---
% E
% ---
%\path (E) edge [bend left] node [below] {$0.3$} (A);

%\path (E) edge [bend left] node [fill=white, anchor=center, pos=0.25] {$0.3$} (B);

%\path (E) edge [bend left] node [right] {$0.0$} (C);

\path (E) edge [bend left] node [fill=white, anchor=center, pos=0.25] {$\frac{1}{3}$} (D);

\path (E) edge [in=348, out=18, loop] node [right] {$\frac{2}{3}$} (E);
\end{tikzpicture} 
\caption{Diagrama de comunicación del reactivo 3} 
\label{fig:p1r3}
\end{figure}
\paragraph{b)} Las clases de comunicación son 
\[\{\{1\},\{2\},\{3\},\{4,5\}\}\]
\paragraph{c)} Veamos que como $2\to 1$, $1\to 5$; $\{1\},\{2\}$ no son clases cerradas. En cambio como $3$ es absorbente, y $4\not\to 1,2,3$ ni $5\to 1,2,3$; las clases de comunicación cerradas son
\[\{3\},\{4,5\}\]
\paragraph{d)} Sólo hay un estado absorbente que es $3$ pues $P(3,3)=1$.
\paragraph{e)} Como la transitoreidad y la absorbencia son propiedades de clase, recordemos que $\{3\},\{4,5\}$ son clases de comunicación cerradas con cantidad de estados finitos y por tanto sus estados son recurrentes. En cambio, las clases restantes no son cerradas y por tanto transitorias. Así los estados \textit{transitorios} son:
\[1,2\]
y los estados \textit{recurrentes} son
\[3,4,5\]
\paragraph{f)} Por Chapman-Kolmogorov, tenemos que
\begin{align*}
    \p(X_2=4|X_0=1)=&\sum_{z\in E}\p(X_2=4|X_1=z)\p(X_1=z|X_0=1)\\
    =&\p(X_2=4|X_1=1)\cancelto{0}{\p(X_1=1|X_0=1)}\\
    &+\p(X_2=4|X_1=2)\cancelto{0}{\p(X_1=2|X_0=1)}\\
    &+\cancelto{0}{\p(X_2=4|X_1=3)}\cancelto{0}{\p(X_1=3|X_0=1)}\\
    &+\p(X_2=4|X_1=4)\p(X_1=4|X_0=1)\\
    &+\p(X_2=4|X_1=5)\p(X_1=5|X_0=1)\\
    =&\p(X_2=4|X_1=4)\p(X_1=4|X_0=1)\\
    &+\p(X_2=4|X_1=5)\p(X_1=5|X_0=1)\\
    =&\left(\frac{1}{2}\right)\left(\frac{1}{2}\right)
    +\left(\frac{1}{3}\right)\left(\frac{1}{2}\right)\\
    =&\frac{1}{4}+\frac{1}{6}\\
    =&\frac{5}{12}
\end{align*}
\subsection{Reactivo 4}
\begin{displayquote}
    Sea $X=X=\cdm[n][0]{X_n}$ una cadena de Markov con espacio de estados $E$ y probabilidades de transición $(P_{x,y})_{x,y\in E}$. Sea $y\in E$ y consideremos como distribución inicial $\delta_y$ (es decir, $X_0=y$) con probabilidad $1$
    \par Supongamos que $y$ es un estado transitorio y consideremos la variable aleatoria
    \[V_y=\sum_{n=0}^\infty\uno(X_n=y)\]
    que corresponde al \textit{número de visitas} al estado $y$. Determian la distribución de $V_y$, indicando el nombre de la distribución y su parámetro.
\end{displayquote}
\section{Problema 2}
\begin{displayquote}
    Una pulga amaestrada brinca aleatoriamente en los vértices de un triángulo, todos sus saltos tienen la misma probabilidad (se supone que en cada etapa cambia de lugar).
    \begin{enumerate}
        \item Calcula la probabilidad de que después de $n$ saltos la pulga regrese a su punto de partida
        \item Suponga que otra pulga brinca en los vértices de un triángulo, pero que esta brinca el doble de veces a su derecha que a su izquierda. ¿Cuál es la probabilidad de que después de $n$ saltos la pulga regrese a su punto de partida?
    \end{enumerate}
\end{displayquote}
\paragraph{a)} Notemos que como en cada etapa cambia de lugar $P(x,x)=0$, de ahí como cada salto tiene la misma probabilidad, así la matriz de transición es
    \[P=\begin{pNiceMatrix}
        0&\frac{1}{2}&\frac{1}{2}\\
        \frac{1}{2}&0&\frac{1}{2}\\
        \frac{1}{2}&\frac{1}{2}&0
    \end{pNiceMatrix}\]

\begin{figure}[!hbt]
\centering
\begin{tikzpicture}[->, >=stealth,shorten >=2pt, line width=0.5pt, node distance=2cm]
% Nodos
\node [circle , draw , text opacity=1, text=black] (A)  at  (90:5cm) {$1$};
\node [circle, fill=gray!10,text opacity=1, text=black] (B)  at (210:5cm) {$2$};
\node [circle, fill=gray!30 , draw, opacity=0.5 ,dotted, text opacity=1, text=black] (C)  at (330:5cm) {$3$};
% Ejes
% ---
% A
% ---
%\path (A) edge [loop above] node [above] {$0.4$} (A);

\path (A) edge node [fill=white, anchor=center, pos=0.25] {$\frac{1}{2}$} (B);

\path (A) edge [bend left] node [fill=white, anchor=center, pos=0.25] {$\frac{1}{2}$} (C);


% ---
% B
% ---
\path (B) edge [bend left] node [fill=white, anchor=center, pos = 0.25] {$\frac{1}{2}$} (A);

%\path (B) edge [in=132, out=162, loop] node [above] {$0.5$} (B);

\path (B) edge node [fill=white, anchor=center, pos = 0.25] {$\frac{1}{2}$} (C);

% ---
% C
% ---
\path (C) edge node [fill=white, anchor=center, pos=0.25] {$\frac{1}{2}$} (A);

\path (C) edge [bend left] node [fill=white, anchor=center, pos=0.25] {$\frac{1}{2}$} (B);

%\path (C) edge [in=204, out=234, loop] node [below] {$1$} (C);

\end{tikzpicture} 
\caption{Diagrama de comunicación del problema 2a} 
\label{fig:p2a}
\end{figure}
Consideremos al diagrama de comunicación (\cref{fig:p2a}). Notemos que la matriz de transición es simétrica, por teorema espectral podemos diagonalizar la matriz. En particular,
\[
P=
\begin{pNiceMatrix}
        -1 & -1 & 1\\1 & 0 & 1\\0 & 1 & 1
\end{pNiceMatrix}
\begin{pNiceMatrix}
    -\frac{1}{2} & 0 & 0\\0 & -\frac{1}{2} & 0\\0 & 0 & 1
\end{pNiceMatrix}
\begin{pNiceMatrix}
    - \frac{1}{3} & \frac{2}{3} & - \frac{1}{3}\\- \frac{1}{3} & - \frac{1}{3} & \frac{2}{3}\\\frac{1}{3} & \frac{1}{3} & \frac{1}{3}
\end{pNiceMatrix}
\]
Entonces 
\[P^n=\begin{pNiceMatrix}
        -1 & -1 & 1\\1 & 0 & 1\\0 & 1 & 1
\end{pNiceMatrix}
\begin{pNiceMatrix}
    -\frac{1}{2} & 0 & 0\\0 & -\frac{1}{2} & 0\\0 & 0 & 1
\end{pNiceMatrix}^n
\begin{pNiceMatrix}
    - \frac{1}{3} & \frac{2}{3} & - \frac{1}{3}\\- \frac{1}{3} & - \frac{1}{3} & \frac{2}{3}\\\frac{1}{3} & \frac{1}{3} & \frac{1}{3}
\end{pNiceMatrix}
=\begin{pNiceMatrix}
    \frac{2 \left(- \frac{1}{2}\right)^{n}}{3} + \frac{1}{3} & \frac{1}{3} - \frac{\left(- \frac{1}{2}\right)^{n}}{3} & \frac{1}{3} - \frac{\left(- \frac{1}{2}\right)^{n}}{3}\\\frac{1}{3} - \frac{\left(- \frac{1}{2}\right)^{n}}{3} & \frac{2 \left(- \frac{1}{2}\right)^{n}}{3} + \frac{1}{3} & \frac{1}{3} - \frac{\left(- \frac{1}{2}\right)^{n}}{3}\\\frac{1}{3} - \frac{\left(- \frac{1}{2}\right)^{n}}{3} & \frac{1}{3} - \frac{\left(- \frac{1}{2}\right)^{n}}{3} & \frac{2 \left(- \frac{1}{2}\right)^{n}}{3} + \frac{1}{3}
\end{pNiceMatrix}
\]
Es entonces que para $x\in E$
\[P^{(n)}(x,x)=\left(\frac{2}{3}\right)\left(-\frac{1}{2}\right)^n+\frac{1}{3}\]

\begin{figure}[!hbt]
\centering
\begin{tikzpicture}[->, >=stealth,shorten >=2pt, line width=0.5pt, node distance=2cm]
% Nodos
\node [circle , draw , text opacity=1, text=black] (A)  at  (90:5cm) {$1$};
\node [circle, fill=gray!10,text opacity=1, text=black] (B)  at (330:5cm) {$2$};
\node [circle, fill=gray!30 , draw, opacity=0.5 ,dotted, text opacity=1, text=black] (C)  at (210:5cm) {$3$};
% Ejes
% ---
% A
% ---
%\path (A) edge [loop above] node [above] {$0.4$} (A);

\path (A) edge [bend left] node [fill=white, anchor=center, pos=0.25] {$\frac{2}{3}$} (B);

\path (A) edge  node [fill=white, anchor=center, pos=0.25] {$\frac{1}{3}$} (C);


% ---
% B
% ---
\path (B) edge  node [fill=white, anchor=center, pos = 0.25] {$\frac{1}{3}$} (A);

%\path (B) edge [in=132, out=162, loop] node [above] {$0.5$} (B);

\path (B) edge [bend left] node [fill=white, anchor=center, pos = 0.25] {$\frac{2}{3}$} (C);

% ---
% C
% ---
\path (C) edge [bend left] node [fill=white, anchor=center, pos=0.25] {$\frac{2}{3}$} (A);

\path (C) edge  node [fill=white, anchor=center, pos=0.25] {$\frac{1}{3}$} (B);

%\path (C) edge [in=204, out=234, loop] node [below] {$1$} (C);

\end{tikzpicture} 
\caption{Diagrama de comunicación del problema 2b} 
\label{fig:p2b}
\end{figure}
    
\paragraph{b)} Notemos del diagrama de comunicación (\cref{fig:p2b}), que la matriz deja de ser simétrica. La matriz de transición queda
\[P=\begin{pNiceMatrix}
    0&\frac{2}{3}&\frac{1}{3}\\
    \frac{1}{3}&0&\frac{2}{3}\\
    \frac{2}{3}&\frac{1}{3}&0\\
\end{pNiceMatrix}\]
Notemos que
\[\chi_P(\lambda)=\lambda^3-\frac{2}{3}\lambda-\frac{1}{3}=(\lambda-1)\left(\lambda^2+\lambda+\frac{1}{3}\right)\]
De donde
\[\lambda_{2,3}=\frac{-1\pm\sqrt{1-\frac{4}{3}}}{2}=-\frac{1}{2}\pm \frac{i}{2\sqrt{3}}=-\frac{1}{\sqrt{3}}\exp\left(\pm i\frac{\pi}{6}\right)\]
De aquí recordando que, como los 3 eigenvalores son diferentes, hay 3 eigenvectores asociados y podemos construir una base a partir de ellos, tales que podemos escribir a $P$ y por ende a $P^n$ en términos de los eigenvalores, por la fórmula de Moivre podemos colapsar los escalares de los 2 términos complejos con dos escalares y así
\[(P^n)_{ij}=a+\left(-\frac{1}{\sqrt{3}}\right)^n\left(b\cos\left(\frac{\pi n}{6}\right)+c\sen\left(\frac{\pi n}{6}\right)\right)\]
Veamos que en el paso cero $P^0$ es la identidad, $P_{1,1}=0$ y 
\[P^2=\begin{pNiceMatrix}
    \frac{4}{9} & \frac{1}{9} & \frac{4}{9}\\\frac{4}{9} & \frac{4}{9} & \frac{1}{9}\\\frac{1}{9} & \frac{4}{9} & \frac{4}{9}
\end{pNiceMatrix}\]
Por lo que $(P^2)_{1,1}=\frac{4}{9}$
Entonces
\begin{align*}
    1=(P^0)_{1,1}&=a+b\\
    0=(P^1)_{1,1}&=a+\left(-\frac{1}{\sqrt{3}}\right)\left(b\cos\left(\frac{\pi}{6}\right)+c\sen\left(\frac{\pi}{6}\right)\right)\quad\\
    \frac{4}{9}=(P^2)_{1,1}&=a+\left(-\frac{1}{\sqrt{3}}\right)^2\left(b\cos\left(\frac{2\pi }{6}\right)+c\sen\left(\frac{2\pi }{6}\right)\right)
\end{align*}
De aquí
\[\systeme{
a+b=1,a-\frac{1}{2}b-\frac{1}{2\sqrt{3}}c=0,
a+\frac{1}{6}b+\frac{\sqrt{3}}{6}c=\frac{4}{9}
}\]
Notemos de aquí que sumando la segunda y tercera ecuación y multiplicando; además de escalar por 3 la primera ecuación nos queda el sistema
\[\systeme{3a+3b=3,18a-3b=4}\]
Lo que nos dice que $a=\frac{1}{3},b=\frac{2}{3},c=0$. Así, sin pérdida de generalidad podemos decir que
\[(P^n)_{x,x}=\frac{1}{3}+\frac{2}{3}\left(-\frac{1}{\sqrt{3}}\right)^n\cos\left(\frac{n\pi}{6}\right)\]
\section{Problema 3}
\begin{displayquote}
    Supongamos que la probabilidad de que llueva hoy es de $0.3$ si ninguno de los últimos dos días fue lluvioso, pero la probabilidad de que llueva hoy es $0.6$ si al menos uno de los últimos dos días fue lluvioso.
    \par Indicamos con $W_n$ el clima del $n-$ésimo día y denotamos
    \[W_n=\begin{cases}
        R& \text{si en el $n$-ésimo día llovió,}\\
        S& \text{si el clima del $n$-ésimo día fue soleado}
    \end{cases}\]
    \begin{enumerate}
        \item Muestra que $W=\cdm{W_n}$ no es una cadena de Markow
        \item Considera el proceso estocástico $X=\cdm{X_n}$ definido por
        \[X_n= (W_{n-1},W_n),\qquad n\geqslant 1\]
        Demuestra que $X$ es una cadena de Markov con espacio de estados 
        \[\{(R,R),(R,S),(S,R),(S,S)\}\]
        \item Calcula la matriz de transición de $X$
        \item Calcula las probabilidades de transición a dos pasos de $X$
        \item ¿Cuál es la probabilidad de que sí llueva un miércoles dado que no llovió el domingo ni el lunes?
    \end{enumerate}
\end{displayquote}
\paragraph{a)} Por construcción, se debería romper desde $n=3$ pues
\[\p(X_3=R|X_2=S,X_1=S)=0.3\qquad \p(X_3=R|X_2=S,X_1=R)=0.6\]
Y como no depende únicamente del estado inmediato anterior, no se cumple con la propiedad de Markov.
\paragraph{b)} Queremos ver que
\begin{align*}
\p(X_n=(W_{n-1},W_{n})|X_{n-1}=(W_{n-2},W_{n-1}),\cdots,X_1=(W_{0},W_{1}))\\
=\p(X_n=(W_{n-1},W_{n})|X_{n-1}=(W_{n-2},W_{n-1}))
\end{align*}
Definamos a $\{A_k\}_{k\geqslant 2}$ como una familia de eventos independientes correspondientes a $\p(W_k=w_k|W_{k-1}=w_{k-1},W_{k-2}=w_{k-2})$. Recordemos también que por construcción $W_k$ depende únicamente de los últimos dos días anteriores. Así,
\begin{align*}
    &\p(X_n=(W_{n-1},W_{n})|X_{n-1}=(W_{n-2},W_{n-1}),\cdots,X_1=(W_{0},W_{1}))\\
    =&\frac{\p(X_n=(W_{n-1},W_{n}),X_{n-1}=(W_{n-2},W_{n-1}),\cdots,X_1=(W_{0},W_{1}))}{\p(X_{n-1}=(W_{n-2},W_{n-1}),\cdots,X_1=(W_{0},W_{1}))}\\
    =&\frac{\p(A_n,\cdots,A_2)}{\p(A_{n-1},\cdots,A_2)}=\p(A_n)\\
    =&\p(W_n=w_n|W_{n-1}=w_{n-1},W_{n-2}=w_{n-2})\\
    =&\p(W_n=w_n,W_{n-1}=w_{n-1}|W_{n-1}=w_{n-1},W_{n-2}=w_{n-2})\\
    =&\p(X_n=(W_{n-1},W_{n})|X_{n-1}=(W_{n-2},W_{n-1}))
\end{align*}
Donde la simplificación de la familia de eventos $X_0,...,X_n$ se da por un proceso análogo a las últimas 3 igualdades a la inversa.
\paragraph{c)} Veamos que 
{\small\begin{align*}
    P[(R,R),(R,R)] &= 0.6 & P[(R,R),(R,S)] &=0.4 & 
    P[(R,R),(S,R)] &= 0.0 & P[(R,R),(S,S)] &=0.0\\
    P[(R,S),(R,R)] &= 0.0 & P[(R,S),(R,S)] &=0.0 & 
    P[(R,S),(S,R)] &= 0.6 & P[(R,S),(S,S)] &=0.4\\
    P[(S,R),(R,R)] &= 0.6 & P[(S,R),(R,S)] &=0.4 & 
    P[(S,R),(S,R)] &= 0.0 & P[(S,R),(S,S)] &=0.0\\
    P[(S,S),(R,R)] &= 0.0 & P[(S,S),(R,S)] &=0.0 & 
    P[(S,S),(S,R)] &= 0.3 & P[(S,S),(S,S)] &=0.7
\end{align*}}
Por lo que
\[
P = \begin{pNiceMatrix}
    0.6&0.4&0.0&0.0\\
    0.0&0.0&0.6&0.4\\
    0.6&0.4&0.0&0.0\\
    0.0&0.0&0.3&0.7
\end{pNiceMatrix}
\]
\paragraph{d)} Por el inciso anterior,
\[P^{(2)}=\begin{pNiceMatrix}
    0.36 & 0.24 & 0.24 & 0.16\\0.36 & 0.24 & 0.12 & 0.28\\0.36 & 0.24 & 0.24 & 0.16\\0.18 & 0.12 & 0.21 & 0.49
\end{pNiceMatrix}\]
\paragraph{e)} Por Chapman-Kolmogorov
\begin{align*}
    &\p(W_3=R|W_1=S,W_0=S)=\p(X_3=(R,R)|X_1=(S,S))+ \p(X_3=(S,R)|X_1=(S,S))\\
    =&\p(X_3=(R,R)|X_2=(S,R))\p(X_2=(S,R)|X_1=(S,S))\\
    &+\p(X_3=(S,R)|X_2=(S,S))\p(X_2=(S,S)|X_1=(S,S))\\
    =&(0.6)(0.3)+(0.3)(0.7)\\
    =&0.39
\end{align*}
\section{Problema 4}
Seis niños (Dick, Helen, Joni, Mark,
Sam y Tony) juegan a atrapar la pelota y sucede lo siguiente:
\begin{itemize}
    \item Si Dick tiene la pelota, es igualmente probable que se la lance a Helen, Mark, Sam y Tony.
    \item Si Helen tiene la pelota, es igualmente probable que se la lance a Dick, Joni, Sam y Tony.
    \item Si Sam tiene la pelota, es igualmente probable que se la lance a Dick, Helen, Mark y Tony.
    \item Si Joni o Tony reciben la pelota, siguen lanzándosela el uno al otro.
    \item Si Mark recibe el balón, se escapa con él.
\end{itemize}
\begin{enumerate}
    \item Encuentra la matriz de transición de la cadena de Markov que modela este juego
    \item Determina las clases de comunicación de la cadena de Markov y determina cuáles clases son cerradas
    \item Para cada estado, determina si es transtiorio o recurrente
    \item Para cada estado, determina si es absorbente
    \item Supongamos que Dick tiene la pelota al comienzo del juego. ¿Cuáles la probabilidad de que Mark termine con la pelota?
\end{enumerate}


\end{document}
